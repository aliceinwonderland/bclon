% LaTeX document
\documentclass[11pt]{article}
\usepackage[spanish,activeacute]{babel}
\usepackage{graphicx}
\usepackage{epsfig}
\usepackage{subfigure}

\setlength{\topmargin}{-.5in}
\setlength{\textheight}{9in}
\setlength{\oddsidemargin}{.125in}
\setlength{\textwidth}{6.25in}

\begin{document}

\title{La otra clonaci'on}
\author{Jamie Drake}

\maketitle
Cuando escuchamos o leemos la palabra clonaci'on, generalmente la asociamos al hecho de obtener una copia gen'etica de una entidad biol'ogica. Sin embargo, existe otra clonaci'on, una t'ecnica que surge en el 'area de inteligencia artificial y cuyo fin es poder copiar las habilidades de una persona. A esa t'ecnica se le conoce como clonaci'on de comportamiento. En este art'iculo se describir'a de qu'e se trata, su origen, sus aplicaciones y perspectivas.


%%%%%%%%%%%%%%%%%%%%%%%%%%%%%%%%%%%%%%%%%%%%%%%%%%%%%%%%
\section{Hab'ia una vez ...}
Las ideas sobre m'aquinas que aprendieran, sobre m'aquinas que tuvieran inteligencia surgieron fuertemente desde los a~nos sesenta. Donald Michie (1923 - 2007), doctor en gen'etica de mam'iferos(1953) fue uno de los precursores del aprendizaje autom'atico y sent'o las bases de muchas de las t'ecnicas usadas en la actualidad. Fue en 1993 cuando nace la t'ecnica a quien Donald Michie llam'o clonaci'on de comportamiento. Pero, antes de continuar, conozcamos m'as sobre c'omo ocurrieron las cosas.

Donald Michie no era un experto en computadoras, en esos tiempos no eran un recurso de la vida diaria como lo son ahora. Ni siquiera exist'ia el estudio de las ciencias de la computaci'on  como tal. Esta ciencia empez'o a ser una disciplina acad'emica entre 1950 y principios de los 1960s y se impart'ia como materia, no como carrera. El primer programa acad'emico en los Estados Unidos se origin'o en la Universidad de Purdue hasta 1962. La formaci'on de Donald Michie se orienta hacia las humanidades y biolog'ia. Aparentemente, su formaci'on no ten'ia relaci'on alguna con las aportaciones que 'el har'ia a lo largo de su vida.

\medskip
\textbf{Vientos de guerra.} Eran fines de los a~nos treinta y el avance de la guerra crec'ia di'a a d'ia. En Marzo de 1938, Hitler orden'o la ocupaci'on de Austria y anunci'o su uni'on y en Mayo del mismo a~no orden'o la ocupaci'on de Checoslovaquia. La amenaza de guerra avanzaba sobre Europa. La Escuela de Cifrado y Codificaci'on del gobierno con sede en Londres necesitaba un sitio seguro. por lo que el gobierno Brit'anico inici'o la b'usqueda.

Bletchley Park se ubica 80 km al noroeste de Londres, en el poblado de Bletchley, Inglaterra. Es una propiedad de aproximadamente 60 acres que en el centro tiene una mansi'on construida con estilos arquitect'onicos variados. Por las condiciones de la propiedad: ten'ia: teletipo y v'ias de comunicaci'on, era el lugar adecuado. Bletchley ser'ia la sede de las comunicaciones del gobierno brit'anico donde Las claves y c'odigos de diversos pa'ises del Eje ser'ian descifradas. En Agosto de 1939 los descifradores de c'odigo llegaron, entre ellos, Alan Turing (1912 - 1954). 

\begin{figure}[h]

\centering
\fbox{
\includegraphics[keepaspectratio,width=2cm]{theimg/bp}
}
\caption[Bletchley Park]{Bletchley Park} 
\label{fig:bp}

\end{figure}

\medskip

Y en 1943 el joven Michie recibi'o una beca del Balliol College en Oxford, era una beca para Estudios Cl'asicos, b'asicamente humanidades. Michie no ten'ia experiencia en m'aquinas ni en matem'aticas. En ese mismo a~no, para contribuir con la campa~na b'elica y convencido de que eso garantizar'ia un lugar en la universidad una vez que la guerra terminara, intent'o inscribirse en un curso intensivo de japon'es para oficiales de inteligencia del cual hab'ia sabido a trav'es de un amigo de la familia que era un oficial de la Oficina de Guerra (War Office). El curso iba a ser impartido en la ciudad de Bedford, a 15 millas de Bletchley. Michie se dirigi'o hacia all'a y se present'o a una entrevista solo para que le dijeran que lo hab'ian informado mal pues el curso estaba lleno y que el siguiente iniciaba hasta el oto~no (era primavera). El oficial lo vi'o muy desanimado y le dijo que hab'ia tambi'en un curso de criptograf'ia que iniciar'ia el Lunes siguiente y que pod'ia interesarle para ocupar su tiempo.

Michie y sus profesores descubrieron inmediatamente que ten'ia un sorprendente talento natural para la materia. Seis semanas despu'es lleg'o un oficial de Bletchley para reclutar a alguien para trabajar en Tunny y Michie fu'e elegido. 

Donald Michie fue asignado a una secci'on en Bletchley Park que se hab'ia establecido en Julio de 1942 bajo la direcci'on del Mayor Ralph Tester quien trabajaba con Tunny usando m'etodos manuales. Al tr'afico de datos encriptados generado por el teletipo alem'an, los brit'anicos le llamaban "Fish" le llamaron a la m'aquina alemana codificadora Lorenz SZ40/42 y a su tr'afico como Tunny. Max Newman, matem'atico quien ten'ia asignada la tarea de automatizar la decodificaci'on de los mensajes requiri'o un asistente, Michie era el que estaba m'as familiarizado con los m'etodos manuales. Michie ten'ia aptitudes para probabilidad, estad'istica y l'ogica booleana y era tanto o m'as h'abil que otros colegas entrenados formalmente.

\medskip
En los 1920s los alemanes construyeron la m'aquina Enigma code, ellos cre'ian que los mensajes codificados (militares y operaciones secretas) no podr'ian decodificarse. La m'aquina parec'ia una m'aquina de escribir y era capaz de realizar millones de c'alculos en milisegundos. Los c'odigos secretos que los controlaban se cambiaban diario.

En 1943 Turing encabez'o un grupo de cient'ificos cuyo objetivo era descifrar los c'odigos de Enigma. Para lograrlo, el equipo desarroll'o una computadora que se componc'ia de 1,500 tubos de vac'io. Esa computadora electromec'anica es Colossus y fue una de las principales herramientas para decodificar mensajes.

\begin{figure}[h]
\begin{center}
\fbox{
\subfigure[Enigma]{\includegraphics[width=2cm,keepaspectratio]{theimg/enigma}}  
\hspace{1cm}
\subfigure[Colossus]{\includegraphics[keepaspectratio,height=2cm]{theimg/colossus}} 
}

\caption{Enigma y Colossus}
  \label{fig:enigmacolossus}
\end{center}
\end{figure} 

Donald Michie quien hizo una significativa contribuci'on al trabajo para descifrar el c'odigo, y sus mejoras a la computadora Colossus redujo significativamente el tiempo necesario para interceptar y decodificar los mensajes, pasando de semanas a unas cuantas horas, un logro m'as all'a de lo que se esperaba lograr.

Michie y Turing se hicieron amigos y durante los juegos de ajedrez que compartieron hablaban sobre las posibilidade de construir una computadora que jugara ajedrez, como parte de un un proyecto mayor sobre la creaci'on de una inteligencia artificial.

En 1946 Michie dej'o Bletchley, las computadoras Colossus fueron destruidas y el trabajo se mantuvo en secreto hasta 1989.

\medskip
\textbf{Despu'es de la guerra}. Donald Michie acept'o la beca en Estudios Cl'asicos pero pronto encontr'o los estudios aburridos as'i que los dej'o y se cambi'o a estudiar anatom'ia humana y fisiolog'ia (1949) y un doctorado en Oxford en gen'etica de mam'iferos (1953), haciendo importantes trabajos en gen'etica de mam'iferos. Hizo aportaciones tambi'en en biolog'ia molecular. A principios de lo sesentas, volvi'o a cambiar ahora a ciencias de la computaci'on y se convirti'o en investigador en inteligencia artificial~\cite{colossusJack}. \textquestiondown Qu'e habr'ia pasado si el curso de japon'es no se llena?

En 1946 Donald Michie asisti'o a la escuela de medicina de Oxford y luego hizo una maestr'ia en Anatom'ia Humana y Fisiolog'ia (1949) y luego un doctorado en Oxon en Mammalian Genetics (1953).

Sin embargo, su atenci'on regres'o a sus discusiones con Turing, en particular el asunto sobre si las computadoras pod'ian ser programadas para aprender de la experiencia.

Sin embargo, el inter'es de Michie por el aprendizaje autom'atico segu'ia latente y estuvo siempre al tanto del trabajo que se hac'iaen el National Physical Laboratory en donde exploraban sobre juegos automatizados similares a las m'aquinas para jugar sobre las que Turing y 'el platicaban en Bletchley. Su inter'es particulas era sobre hacer que las computadoras aprendieran de la experiencia. Sin embargo, DM no ten'ia acceso a computadoras y desarrollaba sus sistemas en papel. Como prueba de concepto, uno de esos sistemas fue MENACE (Matchbox Educable Noughts and Crosses Engine), dise~nado en 1960 usando cajas de cerillos y cuentas de cristal de distintos colores (o semillas). MENACE era capaz de aprender por la experiencia, fu'e uno de los primeros experimentos en aprendizaje por refuerzo y aprendi'o a jugar un juego perfecto de gato (tic-tac-toe). MENACE utilizaba el conceptualmente simple algoritmo de prop'osito general BOXES el cual pod'ia descubrir tambi'en estrategias de control para el problema del p'endulo y el carro.
%In 1960 he built Menace, the Matchbox Educable Noughts and Crosses Engine, a game-playing machine consisting of 300 matchboxes and a collection of glass beads of different colours. Each box had a noughts-and-crosses game position drawn on it, and beads inside to represent possible moves that could be made from that position. Inside each box was a wedge that could trap one bead, and a move was chosen by shaking the matchbox and opening it to show which colour bead had been trapped. Menace was able to learn from experience, as each time a game was played beads were added to the box to reinforce successful moves. 

\begin{figure}[h]

\centering
\fbox{
\includegraphics[keepaspectratio,width=5cm]{theimg/menace}
}
\caption[MENACE]{MENACE} 
\label{fig:menace}

\end{figure}

%Verificar 'esto, 
%The system so impressed the US Office of Naval Research that Michie was invited to Stanford to implement Menace on an IBM computer. On his return to England he persuaded the new Science Research Council to fund research into machine intelligence.

Michie fue invitado a Stanford a implementar MENACE en una computadora IBM y a su regreso a Inglaterra, persuadi'o al Consejo de Investigaci'on y Ciencia fundar la investigaci'on en aprendizaje autom'atico. El programa se estableci'on en 1965 en la Universidad de Edinburgo, Michie fue su primer director. Michie fue designado Profesor de aprendizaje autom'atico en 1967 y la unidad cambi'o su nombre a Departamento de Aprendizaje Autom'atico y Percepci'on.

%Artificial intelligence and machine learning were popular research topics in the 1960s and 1970s, with most researchers confident that the main goals of building intelligent computers that could perceive the outside world and respond appropriately would soon be achieved. As it became clear that they had significantly underestimated the complexity of the problem, attention shifted to specific areas, such as computer vision, robotics, machine translation and expert systems that could deal with limited domains of knowledge.

%Michie’s primary research involved finding ways for machines to extract rules and behaviours from example data, so that they could learn from experience, and he developed the technique of “standard induction”. This was effectively applied in industrial plants, for example at a uranium reprocessing plant in Pennsylvania.

%Aware of the broader applications of his research, Michie developed a commercial version, ExpertEase, to make the process of extracting general rules from human experts more efficient.

%Ver esto de ExpertEase...

El siguiente inter'es fue desarrollar t'ecnicas para extracci'on inductiva de conceptos a partir de ejemplos. 

En 1973 su equipo construy'o a Freddy (FREDERICK), el primer robot que era capaz de aprender al mostrarsele c'omo ensamblar un objeto, usaba visi'on para reconocer componentes.

En 1983 revel'o Expert-Ease, un programa que razonaba y pod'ia generar explicaciones. Extra'ia reglas de humanos expertos. 

Lo que se hizo para ML, sistemas basados en reglas y razonamiento por computadora sentaron las bases de lo que actualmente usamos.

Una de las ideas fundamentales para que una computadora presente un comportamiento inteligente es la capacidad de aprender habilidades ... pero, 
%%%%%%%%%%%%%%%%%%%%%%%%%%%%%%%%%%%%%%%%%%%%%%%%%%%%%%%%
\section{\textquestiondown Qu'e es una habilidad?} 

Cuando queremos aprender algo, primero aprendemos lo b'asico, practicamos, adquirimos nuevo conocimiento y a medida que el conocimiento y la pr'actica aumenta, nuestra habilidad crece y nos hacemos expertos. La adquisici'on de una habilidad consta de dos componentes: el cognitivo y el subcognitivo. La parte cognitiva es el c'omo-hacerlo, es adquirir el conocimiento sobre lo que se quiere aprender. Por ejemplo, si est'as aprendiendo un nuevo idioma, primero aprendes la gram'atica, pronunciaci'on b'asica y expresiones simples. Conforme avanzas, vas adquiriendo nuevo conocimiento y practicando, incrementando tu dominio del lenguaje. En el componente cognitivo se te ense~na algo nuevo, piensas en ello, lo expresas en tus propias palabras, observas c'omo la nueva informaci'on se ajusta a otras cosas que ya sabes; est'as consciente de lo que aprendes.

\medskip
Por otro lado, el componente subcognitivo es la modificaci'on de la habilidad que ocurre con la pr'actica y se da de manera autom'atica. La pr'actica lleva al perfeccionamiento de la habilidad, a dominarla, a ser un experto. Sin embargo, si el conocimiento recibido fue equivocado, la habilidad se perfeccionar'a pero siempre con un resultado defectuoso. Por ejemplo, si en el aprendizaje de un nuevo idioma se ense~n'o mal la pronunciaci'on, la pr'actica se hace de acuerdo a esa mala pronunciaci'on. S'olo adquiriendo conocimiento nuevo que sustituya al equivocado y practicando puede corregir el error. En este caso, 'organos que intervienen en la pronunciaci'on tienen que ajustarse gradualmente al nuevo conocimiento adquirido lo que va m'as all'a del alcance de quien pronuncia. No se le puede ordenar expl'icitamente a los m'usculos c'omo moverse. Esta es la parte subcognitiva, lo que conscientemente no puede modificarse ni explicarse.

\medskip
En la clonaci'on de comportamiento, el aprendizaje autom'atico se usa para crear una descripci'on simb'olica donde la introspecci'on por el humano falla porque la tarea de realiza de forma subconsciente. Las actividades subcognitivas requeridas para realizar una habilidad de alto nivel pueden modelarse de tal forma que la habilidad sea explicable y que el modelo sea operacional. Ser explicable significa que la salida del programa de aprendizaje debe poder ser le'ida y entendida por un humano.

Si para un humano el adquirir una habilidad no siempre es sencillo \textquestiondown c'omo ser'a para una computadora?

%%%%%%%%%%%%%%%%%%%%%%%%%%%%%%%%%%%%%%%%%%%%%%%%%%%%%%%%
\section {Clonaci'on de comportamiento}

Esta t'ecnica se origin'o a principios de los noventa a ra'iz de cr'iticas a la Inteligencia Artificial en el sentido de que no se daba importancia a las habilidades de bajo nivel y s'olo se enfocaba a los procesos cognitivos de alto nivel~\cite{michiesammut:cloning}. El t'ermino \textsf{clonaci'on de comportamiento}, que a lo largo de la tesis referiremos como \textsf{clonaci'on}, fu'e acu~nado por Donald Michie en 1992 y tiene por objetivo extraer conocimiento expl'icito de habilidades de bajo nivel. \textsf{Clonaci'on} busca hacer expl'icita una habilidad y que ese conocimiento sirva para construir controladores que reproduzcan esa habilidad. Al ser dif'icil programar manualmente ese tipo de habilidades, el aprendizaje computacional es de gran ayuda para construir estos sistemas de control.

La t'ecnica b'asica consiste en obtener ejemplos o trazas de las variables involucradas en el proceso cuando un operador o sistema ejecuta la tarea. Estos ejemplos se dan como entrada a algoritmos de aprendizaje computacional que pueden construir modelos, por ejemplo reglas o 'arboles, que al ejecutarse producir'an comportamientos similares a quien gener'o los ejemplos. Los modelos resultantes pueden ser incorporados como programas controladores. A estos programas se les llama ``clones'' porque reproducen la habilidad de una persona o sistema.

Una de las ventajas de esta t'ecnica es que no requiere un modelo matem'atico de control tradicional pues no siempre se cuenta con la informaci'on suficiente o el modelo exacto del proceso. La persona que entrena, 'unicamente necesita ejecutar la habilidad m'as no comprender sus fundamentos te'oricos. En este contexto, los ``clones'' son representaciones simb'olicas de comportamientos de bajo nivel. 

Los principales objetivos de esta t'ecnica son: (i) producir ``clones'' que puedan realizar la tarea de control y (ii) producir ``clones'' que hagan expl'icita la habilidad, que describan lo que el operador hace~\cite{modellingskills}. El primer objetivo es importante porque producir un clon que sirva como controlador facilita la programaci'on al no tenerse que codificar expl'icitamente la tarea. El segundo objetivo permite tener informaci'on expl'icita sobre la estrategia de control llevada a cabo por el operador. 

La formulaci'on original de la \textsf{clonaci'on} es la recuperaci'on de la estrategia de control a partir de trazas y es un mapeo directo de estados a acciones. Una traza est'a formada generalmente por pares (Estado, Clase) donde Estado es un vector de atributos $(x_1,x_2,...)$ y Clase es la acci'on efectuada por el operador en ese estado. Una de las principales limitantes de esta representaci'on es que carece de estructura y es dif'icil mostrar la estrategia del operador de forma descriptiva. En el aspecto de control, los ``clones'' obtenidos no son muy robustos con respecto a cambios en la tarea de control.

Las t'ecnicas de aprendizaje que se han usado m'as frecuentemente en \textsf{clonaci'on} son los \textsf{'arboles de decisi'on}~\cite{quinlan1986}, aunque otras t'ecnicas que reconstruyan funciones a partir de ejemplos pueden ser usadas, como las redes neuronales.
 
\section{Aplicaciones de la clonaci'on}
Las principales aplicaciones de la clonaci'on son en sistemas din'amicos para los que resulta dif'icil programar de forma manual las habilidades para controlarlos. Lo que se busca es reconstruir la estrategia de control que el operador ejecuta subconscientemente para generar controladores autom'aticos y que sean expl'icitos para que se pueda ver c'omo le hacen.

\medskip
\noindent

\textbf{El p'endulo invertido}. Es uno de los ejemplos cl'asicos de control. Consiste en un carro que se desliza sobre un riel de longitud fija. Un poste est'a aseguradol carro de modo que s'olo puede balancearse en una dimensi'on. Solo una fuerza de magnitud fija puede aplicrse hacia la izquierda o derecha. La tarea del aprendizaje es construir una estrategia de control que mantenga el poste en equilibrio y al carro sin golpear los extremos del riel. El problema se trata de evitar la falla m'as que de alcanzar un valor espec'ifico. Consiste en mantener balanceado Se tiene un p'endulo eEl objetivo es extraer un modelo que al ejecutarlo permita mantener el p'endulo en equilibrio. Extraer un modelo entendible es tambi'en importante.~\cite{Suc99skillmodelling}~\cite{And99modellingof}~\cite{freyre}

\medskip
\noindent

\textbf{Aprendiendo a volar.} \textbf{Avi'on:} Tener un piloto autom'atico entrenado con las estrategias de vuelo de pilotos experimentados es una de las principales motivaciones de esta aplicaci'on.\\
\textbf{Objetivo:} La idea es que al obtener trazas de pilotos expertos, el simulador pueda realizar un vuelo de forma autom'atica de forma parecida a como lo hace el piloto. Tambi'en se buscaba obtener el modelo de forma expl'icita para poder entender esa estrategia. Ense~nar al piloto autom'atico c'omo despegar, volar a una distancia y altitud determinada, regresar y aterrizar. Usaron C4.5~\cite{Sammut92learningto}\\
Realizar maniobras en turbulencia~\cite{Isaac03goal-directedlearning} usando una descomposici'on jer'arquica\\
Uso de aprendizaje por refuerzo y representaci'on relacional, incorporando conocimiento del dominio~\cite{Morales04learningto}.

\textbf{C'omo se hizo:} Lo que se hizo fu'e tomar trazas de pilotos expertos, procesarlas con algoritmos de inducci'on de 'arboles. Se obtuvieron modelos que se pod'ian ejecutar en el simulador. \\
\textbf{Resultados:} Una de los usos que se le di'o fue para entrenar a pilotos. Se observ'o que los mejores pilotos no generan buenas trazas de aprendizaje puesto que cometen pocos errores, siendo asi, el algoritmo de aprendizaje no puede inducir cosas como 'qu'e hacer en caso de'. Se generaron mejores controladores cuando los pilotos tuvieron que corregir sus acciones.\\
\textbf{Qui'enes:} Claude Sammut~\cite{Sammut92learningto},Andrew Isaac~\cite{Isaac03goal-directedlearning}, Eduardo F. Morales~\cite{Morales04learningto}.

\textbf{Helic'optero.} Desarrollan m'odulos para la automatizaci'on de m'odulos de bajo nivel~\cite{Buskey03c}. Aprendi'o a seguir una trayectoria deseada dando un peque~no n'umero de demostraciones~\cite{heliCoates}. En todos los casos el helic'optero aut'onomo lo hizo mejor que el piloto.

\begin{figure}[h]
\begin{center}
\fbox{
\subfigure[Pilotos]{\includegraphics[width=6cm,keepaspectratio]{theimg/pilotos}}  
\hspace{1cm}
\subfigure[Avi'on]{\includegraphics[width=6cm,keepaspectratio]{theimg/avion}} 
}

\caption{Avion y Pilotos}
  \label{fig:volar}
\end{center}
\end{figure} 

\begin{figure}[h]

\centering
\fbox{
\includegraphics[keepaspectratio,width=5cm]{theimg/helicopteroAm}
}
\caption[Helic'optero]{Helic'optero} 
\label{fig:helicoptero}

\end{figure}

\paragraph{}
\noindent
\textbf{Gr'uas}. Transportar un contenedor de la orilla a una posici'on objetivo en un barco se requieren dos operaciones: posicionar el carro con la cuerda sobre la posici'on de la carga y enlazarla. Hacer esta maniobra requiere tener medidas de seguridad, exactitud. El tiempo de la transportaci'on debe ser el m'inimo. El aspecto m'as importante es controlar el balanceo de la cuerda. Cuando la carga est'a cerca de la posici'on meta, el balanceo ideal deber'ia ser cero.

Se uso simulador~\cite{DBLP:journals/etai/SucB99}

\begin{figure}[h]

\centering
\fbox{
\includegraphics[keepaspectratio,width=3.5cm]{theimg/crane}
}
\caption[Gr'ua]{Gr'ua} 
\label{fig:grua}

\end{figure}

\paragraph{}
\noindent
\textbf{Soccer}. Los agentes de Robosoccer son generalmente programados manualmente, es decir, son preprogramados. Se hizo una interfaz para que una persona jugara Robosoccer como un video-juego, se registraron sus entradas y salidas despu'es de varios encuentros. Las posibles acciones son: girar a la izquierda, correr r'apido o lento, patear pelota, patear a gol. La entrada es lo que la persona ve en el campo y la salida lo que la persona hizo en esa situaci'on. Posteriormente una t'ecnica de ML (PART) se usa y se extrae un modelo que es el que se usa para controlar el agente. Aprendieron estrategias de juego en un simulador usando WEKA y se las pon'ian a agentes~\cite{soccerAler}. Los comportamientos que se aprendieron son: buscar la pelota, llevar la pelota hacia la meta, anotar en presencia de oponentes. Aunque funcion'o relativamente bien, es demasiado reactivo y no considera cosas que el humano si. Se pretende incorporar aprendizaje de mayor nivel como pasar la pelota, disparo a gol, empujar la pelota, dribleo y modelar a nivel de entrenador.

\begin{figure}[h]

\centering
\fbox{
\includegraphics[keepaspectratio,width=3cm]{theimg/robosoccer}
}
\caption[Robosoccer]{Robosoccer} 
\label{fig:robosoc}

\end{figure}


\paragraph{}
\noindent
\textbf{Rob'otica}. Ejemplo de Claire deste~\cite{deste:cloning}, %navegaci'on Markovito~\cite{trps:icm09,mkvito:ieee}.
%navegaci'on Julio~\cite{julioed},

\begin{figure}[h]
\begin{center}
\fbox{
\subfigure[Maid]{\includegraphics[width=4cm,keepaspectratio]{theimg/robobed}}  
\hspace{1cm}
\subfigure[bed]{\includegraphics[width=3cm,keepaspectratio]{theimg/robomaid}} 
\hspace{1cm}
\subfigure[dishes]{\includegraphics[width=4cm,keepaspectratio]{theimg/robodishes}}
}

\caption{Robot de servicio}
  \label{fig:robotservicio}
\end{center}
\end{figure} 


\medskip
\noindent

\textbf{M'as all'a de la clonaci'on b'asica}
El enfoque b'asico de clonaci'on es obtener trazas, aplicar un algoritmo y obtener el modelo. Sin embargo, esto se puede asi de simple solo cuando se tiene por ejemplo, un conjunto de acciones como clases y as'i. 

Pero qu'e pasa si se quiere aprender en otros niveles, entonces las cosas se complican y hay que aplicar otros enfoques, por ejemplo, la descomposici'on, el shaping. Tambi'en todo puede discretizarse pero generalmente las tareas de control tienen atributos num'ericos asi que hay que abordar eso, que si va a ser usando un vector de atributos fijo o si vamos a usar un aprendizaje relacional como con ILP. Y luego est'a el punto de que si vamos a aprender aprendizaje de metas, aprendizaje de jerarqu'ias y conforme se quiere aprender habilidades m'as complejas hay que utilizar enfoques diferentes.

\section{Beneficios y desventajas}

Desventajas:\\
En las primeras aplicaciones de clonaci'on, los clones generados eran poco robustos, con cualquier ligero cambio ya no funcionaban. Poner otras desventajas...

Ventajas:
\begin{itemize}
\item Funci'on doble: control y descripci'on expl'icita.
\item No es necesario ser experto. No es necesario saber control por ejemplo, para poder inducir un controlador. El usuario se enfoca en obtener trazas.
\item Utilizando representaciones relacionales es posible agregar conocimiento del dominio.
\item El aprendizaje autom'atico de sistemas de control puede ayudar a entender mejor las habilidades subcognitivas que son inaccesiblea para introspecci'on.
\item Puede servir para instrucci'on ya que se muestra expl'icitamente lo que se est'a haciendo.
\end{itemize}

Un aspecto que para algunos es ventaja y para otros desventaja es el hecho de que los ejemplos o trazas deben ser buenos para que el controlador funcione bien. Para m'i punto de vista, es una ventaja que dando ejemplos esto pueda hacerse. Cuando se le ense~na algo ya sea a una mascota o a una persona se le trata de dar los mejores ejemplos posibles, no se le va a ense~ñar mal y luego esperar que lo hagan bien. 

Con respecto a si mismo y a otras t'ecnicas. Beneficio social?
\section{Distribuci'on en el mundo}

\bibliographystyle{plain}
\bibliography{refs}
\end{document} 

% LaTeX document
\documentclass[11pt]{article}
\usepackage[spanish,activeacute]{babel}
\usepackage{graphicx}
\usepackage{epsfig}
\usepackage{subfigure}

\setlength{\topmargin}{-.5in}
\setlength{\textheight}{9in}
\setlength{\oddsidemargin}{.125in}
\setlength{\textwidth}{6.25in}

\begin{document}

\title{La otra clonaci'on}
\author{Jamie Drake}

\maketitle
Cuando escuchamos o leemos la palabra clonaci'on, generalmente la asociamos a la obtenci'on de una copia gen'etica de una entidad biol'ogica. Sin embargo, existe otro tipo de clonaci'on, una t'ecnica que surge en el 'area de inteligencia artificial cuyo fin es copiar las habilidades de una persona para que una computadora pueda reproducirlas. A esa t'ecnica se le conoce como clonaci'on de comportamiento. En este art'iculo se describir'a su origen, de qu'e se trata y sus aplicaciones.


%%%%%%%%%%%%%%%%%%%%%%%%%%%%%%%%%%%%%%%%%%%%%%%%%%%%%%%%
\section{Hab'ia una vez ...}
Las ideas sobre m'aquinas que aprendieran, que tuvieran inteligencia surgieron en los a~nos sesenta. Donald Michie (1923 - 2007), doctor en gen'etica fue uno de los precursores del aprendizaje autom'atico y sent'o las bases de muchas de las t'ecnicas usadas en la actualidad. Fue en 1993 cuando nace la t'ecnica a quien Donald Michie llam'o clonaci'on de comportamiento. Pero, antes de continuar, conozcamos m'as sobre el origen de lo que ahora conocemos como aprendizaje autom'atico (\textit{Machine Learning}) y c'omo de un hecho que aparentemente era negativo, se derivaron los conocimientos para generar la tecnolog'ia actual.

%El primer programa acad'emico en los Estados Unidos se origin'o en la Universidad de Purdue hasta 1962. 

\medskip
\textbf{Vientos de guerra.} Eran fines de los a~nos treinta y el avance de la guerra crec'ia di'a a d'ia. En Marzo de 1938, Hitler orden'o la ocupaci'on de Austria y en Mayo del mismo a~no orden'o la ocupaci'on de Checoslovaquia. La amenaza de guerra avanzaba sobre Europa. El gobierno brit'anico contaba con un centro que se dedicaba a la inteligencia en las comunicaciones; este centro era la Escuela de Cifrado y Codificaci'on con sede en Londres y necesitaba un sitio seguro. El gobierno Brit'anico inici'o la b'usqueda... y lo encontr'o en Bletchley Park.

Bletchley Park es una propiedad que se ubica 80 km al noroeste de Londres, en el poblado de Bletchley, Inglaterra. Es una propiedad de aproximadamente 60 acres; en el centro tiene una mansi'on construida con estilos arquitect'onicos variados. Bletchley Park contaba con v'ias de comunicaci'on y teletipo y por sus condiciones, era el lugar adecuado. Bletchley ser'ia la sede de las comunicaciones del gobierno brit'anico donde las claves y c'odigos de diversos pa'ises del Eje ser'ian descifrados. Y para Agosto de 1939 los descifradores de c'odigo empezaron a llegar, entre ellos, Alan Turing (1912 - 1954). 

\begin{figure}[h]

\centering
\fbox{
\includegraphics[keepaspectratio,width=2cm]{theimg/bp}
}
\caption[Bletchley Park]{Bletchley Park} 
\label{fig:bp}

\end{figure}

\medskip

%Donald Michie no era un experto en computadoras; en esos tiempos, las computadoras no eran un recurso de uso com'un como lo son ahora. Por lo tanto, no exist'ia el estudio de las ciencias de la computaci'on  como tal. Esta ciencia empez'o a ser una disciplina acad'emica entre 1950 y principios de los 1960s. Los intereses de Donald Michie se orientaron inicialmente hacia las humanidades por lo que no imaginaba las aportaciones que 'el har'ia a las ciencias computacionales.

Y mientras eso suced'ia, en 1943 el joven Michie recibi'o una beca para Estudios Cl'asicos del Balliol College en Oxford. En ese mismo a~no, un amigo de la familia que era un oficial de la oficina de guerra le cont'o sobre los cursos de japon'es para oficiales de inteligencia. A Michie le interesaba contribuir con la campa~na b'elica y convencido de que eso le garantizar'ia un lugar en la universidad una vez que la guerra terminara, intent'o inscribirse. El curso iba a ser impartido en la ciudad de Bedford, a 15 millas de Bletchley; era primavera y Michie se dirigi'o hacia all'a para presentarse a una entrevista. Sin embargo, le comunicaron que el curso ya estaba lleno y el siguiente iniciar'ia en el oto~no. El oficial lo vi'o muy desanimado y le plante'o otra opci'on que le pod'ia interesar: un curso de criptograf'ia que iniciar'ia el Lunes siguiente. Las cosas no salieron de acuerdo a sus planes y no imaginaba lo que le esperaba.

Aunque Michie no ten'ia experiencia en m'aquinas ni en matem'aticas, muy pronto sus profesores descubrieron que pose'ia un impresionante talento para las materias. Seis semanas despu'es de iniciar el curso, lleg'o un oficial de Bletchley quien buscaba reclutar a alguien y por los resultados mostrados durante el curso, Michie fue el elegido.

As'i, en poco m'as de un mes de haber llegado, Michie fue asignado a una secci'on en Bletchley Park que se hab'ia establecido en Julio de 1942. En esa secci'on se trabajaba descifrando el tr'afico de datos encriptados generado por el teletipo alem'an al cual los brit'anicos llamaban ``Fish''. Un teletipo es una m'aquina de escribir que se usaba para enviar y recibir mensajes usando el radio y microondas como medio de transmisi'on. Los mensajes del teletipo se encriptaban con una m'aquina alemana codificadora, la Lorenz SZ40/42. Tanto a la m'aquina como a su tr'afico de datos se les llamaba ``Tunny''. Michie inici'o bajo la direcci'on del Mayor Ralph Tester quien trabajaba con ``Tunny'' usando m'etodos manuales. 

Los alemanes transmit'ian cientos de mensajes al d'ia por lo que de no elaborar una estrategia distinta, los m'etodos manuales de desciframiento no ser'ian suficientes. Se necesitaba con urgencia automatizar el proceso. Y fue a Max Newman, un matem'atico te'orico a quien se le asign'o la tarea de automatizar la decodificaci'on de los mensajes. Newman necesitaba un asistente; alguien que estuviera perfectamente familiarizado con los m'etodos manuales, que adem'as tuviera aptitudes para probabilidad, estad'istica y l'ogica booleana. Esa persona no fue otra que Donald Michie. Al grupo se les uni'o Tommy Flowers, ingeniero que por gran su capacidad, impresion'o Alan Turing y lo present'o a Newman. El resultado del trabajo de este equipo fue la computadora electromec'anica Colossus compuesta por 1,500 tubos de vac'io. Colossus fue una de las principales herramientas para decodificar mensajes.

%\medskip
%En los 1920s los alemanes construyeron la m'aquina Enigma code, ellos cre'ian que los mensajes codificados (militares y operaciones secretas) no podr'ian decodificarse. La m'aquina parec'ia una m'aquina de escribir y era capaz de realizar millones de c'alculos en milisegundos. Los c'odigos secretos que los controlaban se cambiaban diario.

%En 1943 Turing encabez'o un grupo de cient'ificos cuyo objetivo era descifrar los c'odigos de Enigma. Para lograrlo, el equipo desarroll'o una computadora que se componc'ia de 1,500 tubos de vac'io. Esa computadora electromec'anica es Colossus y fue una de las principales herramientas para decodificar mensajes.

\begin{figure}[h]
\begin{center}
\fbox{
\subfigure[Enigma]{\includegraphics[width=2cm,keepaspectratio]{theimg/enigma}}  
\hspace{1cm}
\subfigure[Colossus]{\includegraphics[keepaspectratio,height=2cm]{theimg/colossus}} 
}

\caption{Enigma y Colossus}
  \label{fig:enigmacolossus}
\end{center}
\end{figure} 

Donald Michie hizo una significativa contribuci'on al trabajo para descifrar el c'odigo, y sus mejoras a la computadora Colossus redujeron significativamente el tiempo necesario para interceptar y decodificar los mensajes, pasando de semanas a unas cuantas horas, un logro m'as all'a de lo que se esperaba lograr.

En ese tiempo, Michie y Turing se hicieron amigos y durante los juegos de ajedrez que compartieron hablaban sobre las posibilidades de construir una computadora que jugara ajedrez, como parte de un un proyecto mayor sobre la creaci'on de una inteligencia artificial. En 1946 Michie dej'o Bletchley; las computadoras Colossus -diez para entonces- fueron destruidas y el trabajo se mantuvo en secreto hasta 1989.

\medskip
\textbf{Despu'es de la guerra}. Al terminar la guerra, Donald Michie acept'o la beca en Estudios Cl'asicos pero pronto encontr'o los estudios aburridos as'i que los dej'o y opt'o por estudiar anatom'ia humana y fisiolog'ia (1949), posteriormente, obtuvo un doctorado en gen'etica de mam'iferos (1953). Hizo aportaciones tambi'en en biolog'ia molecular. Sin embargo, a principios de los sesenta su atenci'on regres'o a aquellas charlas con Turing durante su estancia en Bletchley. En particular le interesaba investigar si las computadoras pod'ian ser programadas para aprender de la experiencia~\cite{colossusJack}. \textquestiondown Qu'e habr'ia pasado si el curso de japon'es no se llena?

Michie siempre estuvo al tanto del trabajo que se hac'ia en el Laboratorio Nacional de F'isica en donde trabajaban con ideas sobre juegos automatizados similares a las m'aquinas para jugar sobre las que Turing y 'el platicaban en Bletchley. Aunque Michie carec'ia en ese tiempo de acceso a computadoras, ese no fue un obst'aculo para explorar sus ideas. 'El desarrollaba sus sistemas en papel y utilizaba otros recursos. En 1960 dise~n'o MENACE (Matchbox Educable Noughts and Crosses Engine), usando cajas de cerillos (¡si!, ¡cajas de cerillos!) y cuentas de cristal de distintos colores 'o semillas. MENACE era capaz de aprender por la experiencia, fue uno de los primeros experimentos en aprendizaje por refuerzo y aprendi'o a jugar lo que conocemos como gato (tic-tac-toe, noughts and crosses). MENACE utilizaba el conceptualmente simple algoritmo de prop'osito general BOXES el cual pod'ia descubrir tambi'en estrategias de control para el problema del p'endulo y el carro. Actualmente nos quejamos cuando no contamos con acceso a internet; sin embargo, cuando se tiene tal pasi'on por hacer las cosas, no hay obst'aculo v'alido.
%In 1960 he built Menace, the Matchbox Educable Noughts and Crosses Engine, a game-playing machine consisting of 300 matchboxes and a collection of glass beads of different colours. Each box had a noughts-and-crosses game position drawn on it, and beads inside to represent possible moves that could be made from that position. Inside each box was a wedge that could trap one bead, and a move was chosen by shaking the matchbox and opening it to show which colour bead had been trapped. Menace was able to learn from experience, as each time a game was played beads were added to the box to reinforce successful moves. 

\begin{figure}[h]

\centering
\fbox{
\includegraphics[keepaspectratio,width=5cm]{theimg/menace}
}
\caption[MENACE]{MENACE} 
\label{fig:menace}

\end{figure}

%Verificar 'esto, 
%The system so impressed the US Office of Naval Research that Michie was invited to Stanford to implement Menace on an IBM computer. On his return to England he persuaded the new Science Research Council to fund research into machine intelligence.

MENACE caus'o tal inter'es que Michie fue invitado a Stanford a implementarlo en una computadora IBM. A su regreso a Inglaterra, persuadi'o al Consejo de Investigaci'on y Ciencia a fundar la investigaci'on en aprendizaje autom'atico. El programa se estableci'on en 1965 en la Universidad de Edinburgo, donde Michie fue su primer director. En 1967, Michie fue designado Profesor de aprendizaje autom'atico y la unidad cambi'o su nombre a Departamento de Aprendizaje Autom'atico y Percepci'on. Y as'i, esas amenas pl'aticas en Bletchley estaban convirti'endose en realidad.

%Artificial intelligence and machine learning were popular research topics in the 1960s and 1970s, with most researchers confident that the main goals of building intelligent computers that could perceive the outside world and respond appropriately would soon be achieved. As it became clear that they had significantly underestimated the complexity of the problem, attention shifted to specific areas, such as computer vision, robotics, machine translation and expert systems that could deal with limited domains of knowledge.

%Michie’s primary research involved finding ways for machines to extract rules and behaviours from example data, so that they could learn from experience, and he developed the technique of “standard induction”. This was effectively applied in industrial plants, for example at a uranium reprocessing plant in Pennsylvania.

%Aware of the broader applications of his research, Michie developed a commercial version, ExpertEase, to make the process of extracting general rules from human experts more efficient.

%Ver esto de ExpertEase...

El siguiente inter'es de Michie fue desarrollar t'ecnicas para extracci'on inductiva de conceptos a partir de ejemplos. Fue as'i como en 1973 su equipo construy'o a Freddy (FREDERICK), el primer robot que era capaz de aprender al mostrarle c'omo ensamblar un objeto. Freddy usaba visi'on para reconocer componentes. En 1983 desarroll'o Expert-Ease, un programa que razonaba y pod'ia generar explicaciones. El programa extra'ia reglas de humanos expertos.

El objetivo de Michie era lograr que las computadoras mostraran inteligencia y para ello, se requer'ia el aprendizaje de habilidades. Ya no se trataba solamente de apender estrategias de juego sino de la adquisici'on de habilidades que estaban m'as all'a de lo que una persona pod'ia describir. Hacer que una computadora adquiera una habilidad no es algo sencillo; a continuaci'on veremos por qu'e.

%%%%%%%%%%%%%%%%%%%%%%%%%%%%%%%%%%%%%%%%%%%%%%%%%%%%%%%%
\section{\textquestiondown Qu'e es una habilidad?} 

Cuando queremos aprender algo, primero aprendemos lo b'asico, practicamos, adquirimos nuevo conocimiento y a medida que el conocimiento y la pr'actica aumenta, nuestra habilidad crece y nos hacemos expertos. La adquisici'on de una habilidad consta de dos componentes: el cognitivo y el subcognitivo. La parte cognitiva es el c'omo-hacerlo, es adquirir el conocimiento sobre lo que se quiere aprender. Por ejemplo, si se est'a aprendiendo un nuevo idioma, primero se aprende gram'atica, pronunciaci'on b'asica y expresiones simples. Conforme se avanza, se va adquiriendo nuevo conocimiento y practicando, incrementando el dominio del lenguaje. En el componente cognitivo cuando a la persona se le ense~na algo nuevo, piensa en ello, lo expresa en sus propias palabras, observa c'omo la nueva informaci'on se ajusta a otras cosas que la persona ya conoce; la persona es consciente de lo que aprende.

\medskip
Por otro lado, el componente subcognitivo es la modificaci'on de la habilidad que ocurre con la pr'actica y se da de manera autom'atica, sin que la persona sea consciente. Cuando la persona practica, perfecciona la habilidad, y puede llegar a ser un experto. Sin embargo, si el conocimiento recibido fue equivocado, la habilidad se perfeccionar'a pero siempre con un resultado defectuoso. Por ejemplo, si en el aprendizaje de un nuevo idioma se ense~n'o mal la pronunciaci'on, la pr'actica se hace de acuerdo a esa mala pronunciaci'on. S'olo adquiriendo conocimiento nuevo que sustituya al equivocado y practicando puede corregir el error. En este caso, los 'organos que intervienen en la pronunciaci'on tienen que ajustarse gradualmente al nuevo conocimiento adquirido lo que va m'as all'a del alcance de quien pronuncia. No se le puede ordenar expl'icitamente a los m'usculos c'omo moverse. Esta es la parte subcognitiva, lo que conscientemente no puede modificarse ni explicarse.

La pregunta era: \medskip \textquestiondown C'omo hacer que una computadora extrayera o aprendiera una habilidad? Se necesitaba un mecanismo capaz de adquirir el conocimiento a nivel subcognitivo. Con este objetivo, surgi'o la clonaci'on de comportamiento, en la cual, el aprendizaje autom'atico se usa para construir una descripci'on simb'olica donde la introspecci'on por el humano falla porque la tarea de realiza de forma subconsciente. 

%%%%%%%%%%%%%%%%%%%%%%%%%%%%%%%%%%%%%%%%%%%%%%%%%%%%%%%%
\section {Clonaci'on de comportamiento}

Esta t'ecnica se origin'o en parte, motivada por las cr'iticas a la Inteligencia Artificial en el sentido de que no se daba importancia a las habilidades de bajo nivel y s'olo se enfocaba a los procesos cognitivos de alto nivel~\cite{michiesammut:cloning}. El t'ermino \textsf{clonaci'on de comportamiento}, a la que nos referiremos como \textsf{clonaci'on}, fu'e acu~nado por Donald Michie en 1992 y tiene por objetivo extraer conocimiento expl'icito de habilidades de bajo nivel. La \textsf{clonaci'on} busca hacer expl'icita una habilidad y que ese conocimiento sirva para construir controladores que reproduzcan esa habilidad. Al ser dif'icil programar manualmente ese tipo de habilidades, el aprendizaje autom'atico es de gran ayuda para construir estos sistemas de control.

Mediante la clonaci'on, las actividades subcognitivas requeridas para realizar una habilidad de alto nivel pueden modelarse de tal forma que la habilidad sea explicable y que el modelo sea operacional. Ser explicable significa que la salida del programa de aprendizaje debe poder ser le'ida y entendida por un humano.

La t'ecnica b'asica consiste en obtener ejemplos o trazas de las variables involucradas en el proceso cuando un operador o sistema ejecuta la tarea. Los ejemplos se dan como entrada a algoritmos de aprendizaje autom'atico que pueden construir modelos, por ejemplo reglas o 'arboles, que al ejecutarse producir'an comportamientos similares a quien gener'o los ejemplos. Los modelos resultantes pueden ser incorporados como programas controladores. A estos programas se les llama ``clones'' porque reproducen la habilidad de una persona o sistema.

Una de las ventajas de esta t'ecnica es que no requiere un modelo matem'atico de control tradicional pues no siempre se cuenta con la informaci'on suficiente o el modelo exacto del proceso. La persona que entrena o instruye, 'unicamente necesita ejecutar la habilidad m'as no comprender sus fundamentos te'oricos. En este contexto, los ``clones'' son representaciones simb'olicas de comportamientos de bajo nivel. 

Los principales objetivos de esta t'ecnica son: (i) producir ``clones'' que puedan realizar la tarea de control y (ii) producir ``clones'' que hagan expl'icita la habilidad, que describan lo que el operador hace~\cite{modellingskills}. El primer objetivo es importante porque producir un clon que sirva como controlador facilita la programaci'on al no tenerse que codificar expl'icitamente la tarea. El segundo objetivo permite tener informaci'on expl'icita sobre la estrategia de control llevada a cabo por el operador. 

La formulaci'on original de la \textsf{clonaci'on} es la recuperaci'on de la estrategia de control a partir de trazas y es un mapeo directo de estados a acciones. Una traza est'a formada generalmente por pares (Estado, Clase) donde Estado es un vector de atributos $(x_1,x_2,...)$ y Clase es la acci'on efectuada por el operador en ese estado. 

Las t'ecnicas de aprendizaje que se han usado m'as frecuentemente en \textsf{clonaci'on} son los \textsf{'arboles de decisi'on}~\cite{quinlan1986}, aunque otras t'ecnicas que reconstruyan funciones a partir de ejemplos pueden ser usadas, como las redes neuronales.

\medskip
\noindent

\textbf{M'as all'a de la clonaci'on b'asica.}
El enfoque b'asico de clonaci'on consiste en obtener trazas, aplicar un algoritmo y obtener un modelo. Sin embargo, este proceso no es tan directo cuando se tiene una tarea que se compone a su vez de sub-tareas y metas.

Conforme las habilidades a aprender se han ido haciendo m'as complicadas han surgido otros enfoques, por ejemplo, la descomposici'on, el manejo de metas y jerarqu'ias. El tipo de los datos contenidos en las trazas tambi'en es importante. Las tareas de control generalmente contienen atributos num'ericos y aunque es com'un usar la discretizaci'on; existen t'ecnicas que se aplican a este tipo de datos. Otro aspecto importante a determinar es la representaci'on del Estado; si va a ser mediante un vector de atributos fijo o si se va a usar aprendizaje relacional. La clonaci'on se ha combinado tambi'en con otras t'ecnicas de aprendizaje como por ejemplo, aprendizaje por refuerzo. El desarrollo contin'ua y cada vez los retos son mayores.


\section{Aplicaciones de la clonaci'on}
\textquestiondown ¿Acaso no ser'ia muy interesante que se pudiera entrenar a los dispositivos que usamos para que se comporten de acuerdo a nuestras necesidades? Por ejemplo, que el despertador, despu'es de un tiempo de usarlo, sea capaz de activar o desactivar su alarma de acuerdo al entrenamiento adquirido. Y adem'as, que el auto sea capaz de conducirse de forma aut'onoma, con un estilo similar al de su due~no. Estas aplicaciones, no est'an lejos de ser una realidad, si no es que algunas ya existen. Los experimentos realizados con clonaci'on han sido diversos, motivados por esa realidad futura. Veamos a continuaci'on algunos ejemplos:
\medskip
\noindent

\textbf{El poste y el carro}. Es uno de los ejemplos cl'asicos de control y consiste en un carro que se desliza sobre un riel de longitud fija. Un poste est'a asegurado al carro de modo que s'olo puede balancearse en una dimensi'on. El carro se mueve 'unicamente al aplicarle una fuerza de magnitud fija hacia la izquierda o derecha. La tarea del aprendizaje es construir una estrategia de control que mantenga el poste en equilibrio y sin que el carro golpee los extremos del riel. Fue en este problema en el que se mostr'o por primera vez el aprendizaje autom'atico por imitaci'on con entrenamiento de un humano~\cite{Michie94buildingsymbolic,Suc99skillmodelling}. Se han realizado tambi'en experimentos para tutores inteligentes~\cite{And99modellingof}, y en M'exico, se explor'o el control utilizando un enfoque bayesiano~\cite{freyre}.

%\paragraph{}
%\noindent
%\textbf{Operaci'on de gr'uas}. Mover un contenedor desde una posici'on hasta un lugar determinado en un barco es una tarea complicada. Se requiere tener precisi'on para posicionar la cuerda sobre la carga y sujetarla, subirla sin que se caiga. El aspecto m'as importante es controlar el balanceo de la cuerda. Cuando la carga est'a cerca de la posici'on meta, el balanceo ideal deber'ia ser cero.

%Se uso simulador~\cite{DBLP:journals/etai/SucB99}

%\begin{figure}[h]

%\centering
%\fbox{
%\includegraphics[keepaspectratio,width=3.5cm]{theimg/crane}
%}
%\caption[Gr'ua]{Gr'ua} 
%\label{fig:grua}

%\end{figure}

\medskip
\noindent

\textbf{Aprendiendo a volar.} La existencia de simuladores para esta actividad favoreci'o esta aplicaci'on. Existen diversos trabajos sobre este dominio y se han hecho experimentos tanto para aviones como para helic'opteros.\\
Una de las principales motivaciones de esta aplicaci'on ha sido el tener un piloto autom'atico entrenado con las estrategias de vuelo de pilotos experimentados. El objetivo es que al obtener trazas de pilotos expertos, se aprenda un modelo de control para que el simulador pueda realizar un vuelo de forma autom'atica de forma parecida a como lo hace el piloto.  En uno de los trabajos se entren'o al piloto autom'atico con las siguientes tareas: despegar, volar a una distancia y altitud determinada, regresar y aterrizar~\cite{Sammut92learningto}. Se obtuvieron trazas de pilotos expertos, las cuales se procesaron con algoritmos de inducci'on de 'arboles. Los modelos resultantes se pod'ian ejecutar en el simulador. Sorpresivamente, se observ'o que los mejores pilotos no generaron buenas trazas de aprendizaje puesto que comet'ian pocos errores. Las trazas de los pilotos expertos no registraron tantas situaciones de riesgo como para que los algoritmos de aprendizaje pudieran inducir modelos 'utiles. Se generaron mejores controladores cuando los pilotos tuvieron que corregir sus acciones. Este experimento fue aplicable al entrenamiento de pilotos.

Conforme la investigaci'on avanza, se van agregando complicaciones a los experimentos, de modo que los modelos que se aprendan sean m'as realistas. As'i, en experimentos posteriores se aprendi'o a realizar maniobras de turbulencia usando una descomposici'on jer'arquica~\cite{Isaac03goal-directedlearning}. En M'exico, se realizaron experimentos combinando la clonaci'on con aprendizaje por refuerzo y una representaci'on relacional~\cite{Morales04learningto}.\\
En experimentos m'as recientes~\cite{heliCoates}, se utiliz'o un helic'optero miniatura XCell para aprender a seguir una trayectoria deseada proporcionando un peque~no n'umero de demostraciones. En todas las pruebas realizadas, el helic'optero aut'onomo lo hizo mejor que el piloto. %Desarrollan m'odulos para la automatizaci'on de m'odulos de bajo nivel~\cite{Buskey03c}. 

\begin{figure}[h]
\begin{center}
\fbox{
\subfigure[Pilotos]{\includegraphics[width=6cm,keepaspectratio]{theimg/pilotos}}  
\hspace{1cm}
\subfigure[Avi'on]{\includegraphics[width=6cm,keepaspectratio]{theimg/avion}} 
}

\caption{Avion y Pilotos}
  \label{fig:volar}
\end{center}
\end{figure} 

\begin{figure}[h]

\centering
\fbox{
\includegraphics[keepaspectratio,width=5cm]{theimg/helicopteroAm}
}
\caption[Helic'optero]{Helic'optero} 
\label{fig:helicoptero}

\end{figure}

\paragraph{}
\noindent
\textbf{Robosoccer}. Esta aplicaci'on~\cite{soccerAler} surgi'o motivada por RoboCup, una iniciativa cuyo objetivo es fomentar la investigaci'on en inteligencia artificial. RoboCup usa el juego de soccer como dominio de prueba y aunque la idea es desarrollar un equipo de robots humanoides completamente aut'onomos que pueda ganar al equipo humano campe'on del mundo, existen diversas categor'as. Una de ellas es la liga en simulaci'on que usa un software llamado Robosoccer. Los agentes de Robosoccer son generalmente programados manualmente, es decir, son preprogramados. 

El objetivo de esta aplicaci'on es que los agentes aprendan a jugar de forma similar a como lo hace un humano experto. Para lograrlo, se hizo una interfaz para que una persona jugara Robosoccer como si fuera un video-juego. La persona jug'o varios encuentros durante los cuales se registraron las entradas y las salidas. Las posibles acciones durante los encuentros eran: girar a la izquierda, correr r'apido o lento, patear pelota, patear a gol. La entrada es lo que la persona ve en el campo y la salida es la acci'on que la persona ejecut'o en cada situaci'on que se le present'o. Posteriormente, mediante algoritmos de aprendizaje se extrae un modelo que es el que se usa para controlar el agente. Los comportamientos que se aprendieron fueron: buscar la pelota, llevar la pelota hacia la meta y anotar en presencia de oponentes. Los agentes con el comportamiento clonado fueron capaces de enfrentar oponentes y anotar goles de forma parecida a la persona de la que aprendieron. Sin embargo, consideraron que su juego es demasiado reactivo y no considera cosas que el humano si. Se pretende incorporar aprendizaje de mayor nivel como pasar la pelota, empujar la pelota, dribleo y modelar a nivel de entrenador.

\begin{figure}[h]

\centering
\fbox{
\includegraphics[keepaspectratio,width=3cm]{theimg/robosoccer}
}
\caption[Robosoccer]{Robosoccer} 
\label{fig:robosoc}

\end{figure}


\paragraph{}
\noindent
\textbf{Rob'otica}. Clonaci'on en robot real que aprendi'o a seguir un objeto y a evadir obst'aculos, usando C4.5 y redes neuronales. El robot fue guiado usando un joystick, ten'ia sonares y c'amara. El robot reconoc'ia objetos verdes usando segmentaci'on de color. Distintos estilos de control: r'apido, lento. 

En M'exico, en el INAOE (Instituto Nacional de Astrof'isica, 'Optica y Electr'onica) navegaci'on Markovito~\cite{rljulioed, trps:icm09,mkvito:ieee}.


\begin{figure}[h]
\begin{center}
\fbox{
\subfigure[Maid]{\includegraphics[width=4cm,keepaspectratio]{theimg/robobed}}  
\hspace{1cm}
\subfigure[bed]{\includegraphics[width=3cm,keepaspectratio]{theimg/robomaid}} 
\hspace{1cm}
\subfigure[dishes]{\includegraphics[width=4cm,keepaspectratio]{theimg/robodishes}}
}

\caption{Robot de servicio}
  \label{fig:robotservicio}
\end{center}
\end{figure} 


\section{Beneficios y desventajas}

Desventajas:\\
En las primeras aplicaciones de clonaci'on, los clones generados eran poco robustos, con cualquier ligero cambio ya no funcionaban. Poner otras desventajas...

Ventajas:
\begin{itemize}
\item Funci'on doble: control y descripci'on expl'icita.
\item No es necesario ser experto. No es necesario saber control por ejemplo, para poder inducir un controlador. El usuario se enfoca en obtener trazas.
\item Utilizando representaciones relacionales es posible agregar conocimiento del dominio.
\item El aprendizaje autom'atico de sistemas de control puede ayudar a entender mejor las habilidades subcognitivas que son inaccesiblea para introspecci'on.
\item Puede servir para instrucci'on ya que se muestra expl'icitamente lo que se est'a haciendo.
\end{itemize}

\section{Distribuci'on en el mundo}

\begin{figure}[h]

\centering

\includegraphics[keepaspectratio,width=14cm]{theimg/mapbc.eps}

\caption[Clonaci'on de comportamiento en el mundo]{Clonaci'on de comportamiento en el mundo} 
\label{fig:bcmapa}

\end{figure}

\bibliographystyle{plain}
\bibliography{refs}
\end{document} 

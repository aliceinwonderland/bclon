% LaTeX document
\documentclass[11pt]{article}
\usepackage[spanish,activeacute]{babel}

\setlength{\topmargin}{-.5in}
\setlength{\textheight}{9in}
\setlength{\oddsidemargin}{.125in}
\setlength{\textwidth}{6.25in}

\begin{document}

\title{La otra clonaci'on}
\author{Jamie Drake}

\maketitle
Cuando escuchamos o leemos la palabra clonaci'on, generalmente lo asociamos al hecho de obtener una copia gen'etica de una entidad biol'ogica. Existe otra clonaci'on, una t'ecnica que surge en el 'area de inteligencia artificial y cuyo fin es poder copiar las habilidades, a esa t'ecnica se le conoce como clonaci'on de comportamiento. En este art'iculo se describir'a de qu'e se trata, origen, aplicaciones y perspectivas.

%%%%%%%%%%%%%%%%%%%%%%%%%%%%%%%%%%%%%%%%%%%%%%%%%%%%%%%%
\section{\textquestiondown Qu'e es una habilidad?} 

Cuando estudiaba la preparatoria me di cuenta de que carezco de habilidad para el dibujo. Una de las materias que llev'e fue dibujo t'ecnico. Cuando la vi en mi lista de materias imagin'e que ser'ia divertido; no sab'ia lo que me esperaba. El profesor hizo 'enfasis en el dibujo de proyecciones isom'etricas. Nuestro trabajo era dibujar el objeto a partir de las vistas que el profesor dibujaba. Algunos de mis compa~neros ten'ian la habilidad de reconstruir el objeto en la mente al instante y lo plasmaban en el papel de forma veloz y precisa. Yo tardaba horas para tener una aproximaci'on de dudosa exactitud. As'i supe que nunca ser'ia arquitecta, ni dise~nadora gr'afica, ni algo parecido. Estaba en un problema, necesitaba salir bien en la materia. Lo que ten'ia que hacer era saber c'omo hacerlo y practicar. \textquestiondown Lograr'ia aprender a hacer los dibujos? \textquestiondown C'omo se adquiere una habilidad?

\medskip

Cuando queremos aprender algo, primero aprendemos lo b'asico, practicamos, adquirimos nuevo conocimiento y a medida que el conocimiento y la pr'actica aumenta, nuestra habilidad crece y nos hacemos expertos. La adquisici'on de una habilidad consta de dos componentes: el cognitivo y el subcognitivo. La parte cognitiva es el c'omo-hacerlo, es adquirir el conocimiento sobre lo que se quiere aprender. Por ejemplo, si est'as aprendiendo un nuevo idioma, primero aprendes la gram'atica, pronunciaci'on b'asica y expresiones simples. Conforme avanzas, vas adquiriendo nuevo conocimiento y practicando, incrementando tu dominio del lenguaje. En el componente cognitivo se te ense~na algo nuevo, piensas en ello, lo expresas en tus propias palabras, observas c'omo la nueva informaci'on se ajusta a otras cosas que ya sabes; est'as consciente de lo que aprendes.

\medskip
Por otro lado, el componente subcognitivo es la modificaci'on de la habilidad que ocurre con la pr'actica y se da de manera autom'atica. La pr'actica lleva al perfeccionamiento de la habilidad, a dominarla, a ser un experto. Sin embargo, si el conocimiento recibido fue equivocado, la habilidad se perfeccionar'a pero siempre con un resultado defectuoso. Por ejemplo, si en el aprendizaje de un nuevo idioma se ense~n'o mal la pronunciaci'on, la pr'actica se hace de acuerdo a esa mala pronunciaci'on. S'olo adquiriendo conocimiento nuevo que sustituya al equivocado y practicando puede corregir el error. En este caso, 'organos que intervienen en la pronunciaci'on tienen que ajustarse gradualmente al nuevo conocimiento adquirido lo que va m'as all'a del alcance de quien pronuncia. No se le puede ordenar expl'icitamente a los m'usculos c'omo moverse. Esta es la parte subcognitiva, lo que conscientemente no puede modificarse ni explicarse.

\medskip
En ese momento yo no estaba pensando en el proceso de adquirir la habilidad. Trat'e de entender las clases, cosa que no me di'o mucho resultado. Consult'e material adicional y resolv'i ejercicios por mi cuenta hasta que logr'e hacer los trabajos en tiempo razonable. Vaya, al principio la regla T se me mov'ia, me sal'ian l'ineas dobles y ten'ia que repetir el trabajo. Poco a poco, el uso de la regla T era autom'atico y no ten'ia que hacer el dibujo primero a l'apiz y luego con el estil'ografo. A falta de una habilidad natural para el dibujo, tuve que dedicarle mucho tiempo a aprender y a practicar. No logr'e ser una experta, pero pude aprobar decorosamente la materia. 

\medskip
Si eso pasa con un humano, imag'inate c'omo es hacer un programa que adquiera habilidades.

\medskip
En la clonaci'on de comportamiento, el aprendizaje autom'atico se usa para crear una descripci'on simb'olica donde la introspecci'on por el humano falla porque la tarea de realiza de forma subconsciente. Las actividades subcognitivas requeridas para realizar una habilidad de alto nivel pueden modelarse de tal forma que la habilidad sea explicable y que el modelo sea operacional. Ser explicable significa que la salida del programa de aprendizaje debe poder ser le'ida y entendida por un humano.

%%%%%%%%%%%%%%%%%%%%%%%%%%%%%%%%%%%%%%%%%%%%%%%%%%%%%%%%
\section {T'ecnicas para aprendizaje autom'atico de habilidades}

\subsection{Clonaci'on de comportamiento}

Esta t'ecnica se origin'o a principios de los noventa a ra'iz de cr'iticas a la Inteligencia Artificial en el sentido de que no se daba importancia a las habilidades de bajo nivel y s'olo se enfocaba a los procesos cognitivos de alto nivel~\cite{michiesammut:cloning}. El t'ermino \textsf{clonaci'on de comportamiento}, que a lo largo de la tesis referiremos como \textsf{clonaci'on}, fu'e acu~nado por Donald Michie en 1992 y tiene por objetivo extraer conocimiento expl'icito de habilidades de bajo nivel. \textsf{Clonaci'on} busca hacer expl'icita una habilidad y que ese conocimiento sirva para construir controladores que reproduzcan esa habilidad. Al ser dif'icil programar manualmente ese tipo de habilidades, el aprendizaje computacional es de gran ayuda para construir estos sistemas de control.

La t'ecnica b'asica consiste en obtener ejemplos o trazas de las variables involucradas en el proceso cuando un operador o sistema ejecuta la tarea. Estos ejemplos se dan como entrada a algoritmos de aprendizaje computacional que pueden construir modelos, por ejemplo reglas o 'arboles, que al ejecutarse producir'an comportamientos similares a quien gener'o los ejemplos. Los modelos resultantes pueden ser incorporados como programas controladores. A estos programas se les llama ``clones'' porque reproducen la habilidad de una persona o sistema.

Una de las ventajas de esta t'ecnica es que no requiere un modelo matem'atico de control tradicional pues no siempre se cuenta con la informaci'on suficiente o el modelo exacto del proceso. La persona que entrena, 'unicamente necesita ejecutar la habilidad m'as no comprender sus fundamentos te'oricos. En este contexto, los ``clones'' son representaciones simb'olicas de comportamientos de bajo nivel. 

Los principales objetivos de esta t'ecnica son: (i) producir ``clones'' que puedan realizar la tarea de control y (ii) producir ``clones'' que hagan expl'icita la habilidad, que describan lo que el operador hace~\cite{modellingskills}. El primer objetivo es importante porque producir un clon que sirva como controlador facilita la programaci'on al no tenerse que codificar expl'icitamente la tarea. El segundo objetivo permite tener informaci'on expl'icita sobre la estrategia de control llevada a cabo por el operador. En esta tesis nos interesa cubrir los dos objetivos puesto que buscamos aprender ``clones'' que sean confiables para el control de un robot m'ovil y tambi'en queremos que los modelos que se obtengan sean f'aciles de interpretar y que puedan utilizarse en otras tareas.  

La formulaci'on original de la \textsf{clonaci'on} es la recuperaci'on de la estrategia de control a partir de trazas y es un mapeo directo de estados a acciones. Una traza est'a formada generalmente por pares (Estado, Clase) donde Estado es un vector de atributos $(x_1,x_2,...)$ y Clase es la acci'on efectuada por el operador en ese estado. Una de las principales limitantes de esta representaci'on es que carece de estructura y es dif'icil mostrar la estrategia del operador de forma descriptiva. En el aspecto de control, los ``clones'' obtenidos no son muy robustos con respecto a cambios en la tarea de control.

Las t'ecnicas de aprendizaje que se han usado m'as frecuentemente en \textsf{clonaci'on} son los \textsf{'arboles de decisi'on}~\cite{quinlan1986}, aunque otras t'ecnicas que reconstruyan funciones a partir de ejemplos pueden ser usadas, como las redes neuronales.
 

\subsection{Aprendizaje por imitaci'on}
 El \textsf{aprendizaje por imitaci'on}, tambi'en conocido como \textsf{programaci'on por demostraci'on}~\cite{aude:pbd} surgi'o a principios de los ochentas como camino para automatizar la programaci'on manual de robots industriales. Las demostraciones se hac'ian por tele-operaci'on, es decir, gu'iando al robot a distancia y las secuencias estado-acci'on se registraban. La informaci'on de los sensores era segmentada en sub-metas discretas que correspond'ian a puntos a lo largo de la trayectoria. Las acciones b'asicas o primitivas eran generalmente movimientos punto-punto para alcanzar las submetas. As'i, la tarea demostrada era segmentada en una secuencia de transiciones estado-acci'on-estado. Esta secuencia era convertida en reglas \textit{if-then} que describ'ian los estados y las acciones de acuerdo a relaciones simb'olicas, tales como ``cerca-de'', ``mover-hacia'', ``mover-sobre''. A estos s'imbolos se le asociaban rangos num'ericos que se daban como conocimiento \textit{a priori} al sistema.  Una demostraci'on completa se representaba en un grafo donde cada nodo representaba un estado y los arcos representaban las acciones. 

Posteriormente se sugiri'o el uso de t'ecnicas de aprendizaje computacional y surgieron as'i preguntas como: ?`c'omo generalizar una tarea?, ?`c'omo reproducir una habilidad en situaciones desconocidas?, ?`c'omo evaluar un intento?, y ?`c'omo definir el papel del usuario durante el aprendizaje?

En los trabajos iniciales se hizo notar la importancia de proporcionar un conjunto de ejemplos numeroso y diverso. Se incorpor'o al usuario como parte activa del aprendizaje al proporcionarle ejemplos que no hubieran sido cubiertos. Actualmente el trabajo que se realiza es muy similar a los iniciales. Se ha avanzado en la forma de guiar al robot, la forma de interacci'on y el tipo de sensores. Se incorporaron t'ecnicas para abordar aspectos como la generalizaci'on entre demostraciones y la generalizaci'on del movimiento en distintas situaciones. Ejemplos de las t'ecnicas son: redes neuronales, l'ogica difusa y modelos ocultos de Markov. La tendencia en el \textsf{aprendizaje por imitaci'on} es basarse en el comportamiento animal y de infantes. La ``bio-inspiraci'on'', como se le llama en el 'area tiene como uno de sus principales componentes la representaci'on sensorimotora. Actualmente la tendencia es que los robots presenten gran flexibilidad tanto en aprendizaje como en interacci'on con humanos. 

%\medskip
Uno de los problemas que caracteriza al \textsf{aprendizaje por imitaci'on} es el problema de correspondencia, que consiste en el mapeo entre el demostrador y el imitador. Si ambos tienen cuerpos similares, entonces una obvia correspondencia es mapear las correspondientes partes del cuerpo: brazo izquierdo del demostrador con brazo izquierdo del imitador, ojo derecho del demostrador con ojo derecho del imitador. En el caso anterior hay una clara correspondencia para la ejecuci'on de las acciones. Una correspondencia no tiene que ser una relaci'on uno-a-uno, puede ser uno-muchos, muchos-uno o muchos-muchos. Aunque el n'umero de grados de libertad (o por sus siglas DOFs del ingl'es \textit{Degrees of Freedom}) en las articulaciones de los brazos de un robot sean distintas puede haber correspondencia. Un robot puede imitar a un humano a saludar sin que el n'umero de sus articulaciones sea la misma. El problema de correspondencia trata de encontrar la secuencia de acciones realizadas por el imitador que correspondan a las del demostrador de acuerdo a su propio cuerpo.

En trabajos recientes~\cite{saunders:self} una propuesta de soluci'on al problema de correspondencia consiste en generar los ejemplos con el mismo robot. A esta t'ecnica le llaman ``auto-imitaci'on''. Sin embargo, esta t'ecnica tiene las mismas caracter'isticas de ``clonaci'on de comportamiento'' que es la que utilizamos en esta tesis y que a continuaci'on se describe.

\textsf{Aprendizaje por imitaci'on} y \textsf{clonaci'on} son t'ecnicas que tienen objetivos y aplicaciones diferentes. Imitaci'on surgi'o con el fin de automatizar la programaci'on de robots industriales. La idea principal es que exista un demostrador que realice la tarea que el imitador va a aprender. El demostrador puede ser diferente al imitador. \textsf{Clonaci'on} busca extraer conocimiento expl'icito de una habilidad para construir controladores que la puedan reproducir. No se origin'o espec'ificamente para rob'otica sino que ha sido aplicada en diversos dominios como por ejemplo, aprendizaje de vuelo, control de gr'uas. La Tabla~\ref{tab:imiclon} resume las diferencias entre ambas t'ecnicas.

\begin{table}[h]
\begin{center}
\begin{small}
\begin{tabular}{|l|l|l|} 

\hline
%\noalign{\smallskip}
& \textsf{Imitaci'on}  & \textsf{Clonaci'on}\\
%\noalign{\smallskip} 
\hline
Inicios & Principios de los 80s.  &    Principios de los 90s.\\ \hline
Objetivo& Automatizar la programaci'on  & (1) Extraer conocimiento expl'icito de habilidades.\\ 
				& manual de robots industriales. & (2) Construir controladores que reproduzcan\\ 
				& 															 & \hspace{0.5cm} la habilidad.\\ \hline
Obtenci'on 		& 	El demostrador puede ser 		 	&  El usuario gu'ia al aprendiz (imitador).\\ 
de ejemplos		& 	diferente al imitador.		 		&  \\ \hline
Aplicaci'on 		& 	Rob'otica. 		 	&  Aprendizaje de vuelo, control de gr'uas,\\ 
											 		& 								 	&  control de bicicleta, veh'iculos, robots, etc. \\ 
\hline

\end{tabular}
\end{small}
\end{center}
\caption[Diferencias entre \textsf{imitaci'on} y \textsf{clonaci'on}]{Diferencias entre \textsf{aprendizaje por imitaci'on} y \textsf{clonaci'on}}
\label{tab:imiclon}
\end{table} 

\section{Aplicaciones de la clonaci'on}
Las principales aplicaciones de la clonaci'on son en sistemas din'amicos para los que resulta dif'icil programar de forma manual las habilidades para controlarlos. Lo que se busca es reconstruir la estrategia de control que el operador ejecuta subconscientemente para generar controladores autom'aticos y que sean expl'icitos para que se pueda ver c'omo le hacen.


\begin{itemize}
\item \textbf{Aprendiendo a volar.} Tener un piloto autom'atico entrenado con las estrategias de vuelo de pilotos experimentados es una de las principales motivaciones de esta aplicaci'on.\\
\textbf{Objetivo:} La idea es que al obtener trazas de pilotos expertos, el simulador pueda realizar un vuelo de forma autom'atica de forma parecida a como lo hace el piloto. Tambi'en se buscaba obtener el modelo de forma expl'icita para poder entender esa estrategia. Ense~nar al piloto autom'atico c'omo despegar, volar a una distancia y altitud determinada, regresar y aterrizar. Usaron C4.5~\cite{Sammut92learningto}\\
Realizar maniobras en turbulencia~\cite{Isaac03goal-directedlearning} usando una descomposici'on jer'arquica\\
Uso de aprendizaje por refuerzo y representaci'on relacional, incorporando conocimiento del dominio~\cite{Morales04learningto}.


\textbf{C'omo se hizo:} Lo que se hizo fu'e tomar trazas de pilotos expertos, procesarlas con algoritmos de inducci'on de 'arboles. Se obtuvieron modelos que se pod'ian ejecutar en el simulador. \\
\textbf{Resultados:} Una de los usos que se le di'o fue para entrenar a pilotos. Se observ'o que los mejores pilotos no generan buenas trazas de aprendizaje puesto que cometen pocos errores, siendo asi, el algoritmo de aprendizaje no puede inducir cosas como 'qu'e hacer en caso de'. Se generaron mejores controladores cuando los pilotos tuvieron que corregir sus acciones.\\
\textbf{Qui'enes:} Claude Sammut~\cite{Sammut92learningto},Andrew Isaac~\cite{Isaac03goal-directedlearning}, Eduardo F. Morales~\cite{Morales04learningto}.
\item El p'endulo invertido. Es uno de los ejemplos cl'asicos de control. Se tiene un p'endulo eEl objetivo es extraer un modelo que al ejecutarlo permita mantener el p'endulo en equilibrio. Extraer un modelo entendible es tambi'en importante.~\cite{Suc99skillmodelling}~\cite{And99modellingof}
\item Gr'uas. Mediante trazas de operadores de gr'ua se aprende un conjunto de ecuaciones usando un sistema llamado Goldhorn
\item Bicicletas. La idea era aprenderse a equilibrar en una bicicleta y conducir a un punto. Se hizo usando aprendizaje por refuerzo y algo llamado shaping.
\item Soccer. Aprendieron estrategias de juego en un simulador usando WEKA y se las pon'ian a agentes.
\item Dolly el helic'optero. Aprendi'o estrategias de control de vuelo mediante aprendizaje por refuerzo.
\item Rob'otica. Ejemplo de Claire deste.
\end{itemize}


\section{M'as all'a de la clonaci'on b'asica}
El enfoque b'asico de clonaci'on es obtener trazas, aplicar un algoritmo y obtener el modelo. Sin embargo, esto se puede asi de simple solo cuando se tiene por ejemplo, un conjunto de acciones como clases y as'i. 

Pero qu'e pasa si se quiere aprender en otros niveles, entonces las cosas se complican y hay que aplicar otros enfoques, por ejemplo, la descomposici'on, el shaping. Tambi'en todo puede discretizarse pero generalmente las tareas de control tienen atributos num'ericos asi que hay que abordar eso, que si va a ser usando un vector de atributos fijo o si vamos a usar un aprendizaje relacional como con ILP. Y luego est'a el punto de que si vamos a aprender aprendizaje de metas, aprendizaje de jerarqu'ias y conforme se quiere aprender habilidades m'as complejas hay que utilizar enfoques diferentes.

\section{Beneficios y desventajas}

Desventajas:\\
En las primeras aplicaciones de clonaci'on, los clones generados eran poco robustos, con cualquier ligero cambio ya no funcionaban. Poner otras desventajas...

Ventajas:
\begin{itemize}
\item Funci'on doble: control y descripci'on expl'icita.
\item No es necesario ser experto. No es necesario saber control por ejemplo, para poder inducir un controlador. El usuario se enfoca en obtener trazas.
\item Utilizando representaciones relacionales es posible agregar conocimiento del dominio.
\item El aprendizaje autom'atico de sistemas de control puede ayudar a entender mejor las habilidades subcognitivas que son inaccesiblea para introspecci'on.
\item Puede servir para instrucci'on ya que se muestra expl'icitamente lo que se est'a haciendo.
\end{itemize}

Un aspecto que para algunos es ventaja y para otros desventaja es el hecho de que los ejemplos o trazas deben ser buenos para que el controlador funcione bien. Para m'i punto de vista, es una ventaja que dando ejemplos esto pueda hacerse. Cuando se le ense~na algo ya sea a una mascota o a una persona se le trata de dar los mejores ejemplos posibles, no se le va a ense~ñar mal y luego esperar que lo hagan bien. 

Con respecto a si mismo y a otras t'ecnicas. Beneficio social?
\section{En rob'otica: el paso a paso}
\section{Perspectivas futuras}

\bibliographystyle{plain}
\bibliography{refs}
\end{document} 
